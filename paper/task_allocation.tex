\documentclass[sigconf]{aamas}  % do not change this line!
\usepackage{balance}  % do not change this line -- unless you manually balance your last page

\usepackage{booktabs}
\usepackage{amsmath}

%% do not change the following lines
\setcopyright{ifaamas}  % do not change this line!
\acmConference[AAMAS'19]{Proc.\@ of the 18th International Conference on Autonomous Agents and Multiagent Systems (AAMAS 2019)}{May 13--17, 2019}{Montreal, Canada}{N.~Agmon, M.~E.~Taylor, E.~Elkind, M.~Veloso (eds.)}  % do not change this line!
\acmYear{2019}  % do not change this line!
\copyrightyear{2019}  % do not change this line!

%% the rest of your preamble here

\settopmatter{printacmref=true}
  % mandatory for ACM publications, do not delete

\fancyhead{}
  % do not delete this code.

\usepackage{balance}
  % for creating a balanced last page (usually end of the references)

%% the rest of your preamble here

\newenvironment{psmallmatrix}
  {\bigg(\begin{smallmatrix}}
  {\end{smallmatrix}\bigg)}
  

%%%%%%%%%%%%%%%%%%%%%%%%%%%%%%%%%%%%%%%%%%%%%%%%%%%%%%%%%%%%%%%%%%%%%%%%%%%%%%%%%%%%%%%%%%%%%%%%%%%%%%%%%

\begin{document}

\title{Emergence of Task Specialisation in Homogeneous versus Heterogeneous Teams}  

 \author{Mostafa Rizk}
 \affiliation{%
  \institution{Faculty of Information Technology, Monash University}
  \city{Melbourne} 
  \state{Australia} 
 }
 \email{mostafa.rizk@monash.edu}

 \author{Julian Garc\'ia}
 \affiliation{%
  \institution{Faculty of Information Technology, Monash University}
  \city{Melbourne} 
  \state{Australia} 
 }
 \email{julian.garcia@monash.edu}

 \author{Aldeida Aleti}
 \affiliation{%
  \institution{Faculty of Information Technology, Monash University}
  \city{Melbourne} 
  \state{Australia} 
 }
 \email{aldeida.aleti@monash.edu}
 
  \author{Giuseppe Cuccu}
 \affiliation{%
 	\institution{eXascale Infolab, Department of Computer Science, University of Fribourg}
 	\city{Fribourg} 
 	\state{Switzerland} 
 }
 \email{giuseppe.cuccu@unifr.ch}

%% The example's default list of authors is too long for headers
%\renewcommand{\shortauthors}{B. Trovato et al.}


\begin{abstract}  % put your abstract here!

Foraging is a common task in the robotics domain that involves transporting resources from one location to another. It has analogues to many real world applications such as mining, space exploration and hazardous waste removal. Using a distributed multi-robot team is advantageous as the team is scalable and robust, but designing controllers is difficult. This is especially true when the team divides the task into different sub-tasks and sub-groups need to specialise. Evolutionary algorithms are a promising tool for controller design, however little work has been done evolving controllers for teams performing task specialisation. This paper examines a foraging scenario requiring task specialisation and explores the difference in performance of a heterogeneous and homogeneous team. In the heterogeneous setup, the sub-groups have different controllers and evolution uses individual selection. In the homogeneous setup, the sub-groups have the same controllers but perform different tasks depending on environmental cues, and evolution uses team selection. We provide results suggesting how to most effectively evolve a team that achieves high performane in this variant of the foraging problem that requires specialisation.

\end{abstract}


% AAMAS: the ACM CCS are not needed within AAMAS papers
%% The code below should be generated by the tool at
%% http://dl.acm.org/ccs.cfm
%% Please copy and paste the code instead of the example below. 
\begin{CCSXML}
<ccs2012>
<concept>
<concept_id>10010147.10010178.10010219.10010220</concept_id>
<concept_desc>Computing methodologies~Multi-agent systems</concept_desc>
<concept_significance>500</concept_significance>
</concept>
<concept>
<concept_id>10010147.10010178.10010219.10010223</concept_id>
<concept_desc>Computing methodologies~Cooperation and coordination</concept_desc>
<concept_significance>500</concept_significance>
</concept>
<concept>
<concept_id>10010147.10010341</concept_id>
<concept_desc>Computing methodologies~Modeling and simulation</concept_desc>
<concept_significance>500</concept_significance>
</concept>
</ccs2012>
\end{CCSXML}

\ccsdesc[500]{Computing methodologies~Multi-agent systems}
\ccsdesc[500]{Computing methodologies~Cooperation and coordination}
\ccsdesc[500]{Computing methodologies~Modeling and simulation}


\keywords{Evolutionary algorithms; Cooperation; Division of labour}  % put your semicolon-separated keywords here!

\maketitle


%%%%%%%%%%%%%%%%%%%%%%%%%%%%%%%%%%%%%%%%%%%%%%%%%%%%%%%%%%%%%%%%%%%%%%%%%%%%%%%%%%%%%%%%%%%%%%%%%%%%%%%%%
%% start of main body of paper

\section{Introduction}



\section{Methods \& Materials \label{preliminaries}}

Hypothesis: The larger the theta, the more likely the heterogeneous setup is to outperform the homogeneous setup



\section{Results\label{results}}
 
\section{Discussion\label{discussion}}



\bibliographystyle{ACM-Reference-Format}  % do not change this line!
\balance  % do not change this line -- unless you manually balance your last page
\bibliography{references,et}  % put name of your .bib file here

\end{document}