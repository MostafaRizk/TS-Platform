\documentclass[12pt]{article}
\title{Purgatory}
\author{Mostafa Rizk}

\begin{document}
\maketitle

 \author{Mostafa Rizk}
 \affiliation{%
  \institution{Faculty of Information Technology, Monash University}
  \city{Melbourne} 
  \state{Australia} 
 }
 \email{mostafa.rizk@monash.edu}

 \author{Julian Garc\'ia}
 \affiliation{%
  \institution{Faculty of Information Technology, Monash University}
  \city{Melbourne} 
  \state{Australia} 
 }
 \email{julian.garcia@monash.edu}

 \author{Aldeida Aleti}
 \affiliation{%
  \institution{Faculty of Information Technology, Monash University}
  \city{Melbourne} 
  \state{Australia} 
 }
 \email{aldeida.aleti@monash.edu}
 
  \author{Giuseppe Cuccu}
 \affiliation{%
 	\institution{eXascale Infolab, Department of Computer Science, University of Fribourg}
 	\city{Fribourg} 
 	\state{Switzerland} 
 }
 \email{giuseppe.cuccu@unifr.ch}
 
 \keywords{Evolutionary algorithms; Cooperation; Division of labour}  % put your semicolon-separated keywords here!

\section{Stuff}

\subsubsection{Heterogeneous Team with Team Selection}

A genome is the weights for two neural networks (288 x 2 weights).
Half of the team loads the first half of the genome while half of the team loads the second half. 
There are 40 teams so 40 genomes are created. 
All members of the team share the reward and cost and a fitness value is returned for the whole team. 
40 fitness values are returned for the whole generation. 
The genome used by the best performing team is selected.\\

\subsubsection{Homogeneous Team with Individual Selection}

A genome is the weights for one individual (288 weights). 
There are 40 teams with two agents per team so 80 genomes are created where every second genome is a duplicate of the previous genome i.e. genome 2 is a copy of genome 1, genome 4 is a copy of genome 3 etc. 
For every 2 (identical) genomes, half the robots on a team get the first and the other half get the second. 
All members of the team share the reward but bear the costs separately. 
A fitness value is returned for each individual so 80 fitness values are returned for the whole generation. 
The genome used by the best performing individual is selected.\\. 

A previous study by Waibel et al suggests that Het-Ind and Hom-Team are the most commonly used configurations.
The study also finds that Hom-Team and Hom-Ind perform equally well, usually, and Het-Ind is sometimes on par but often worse. 
Het-Team performs poorly across the board.
However, this study assumed a task that doesn’t benefit from specialisation. 
We believe that for a task that benefits from specialisation, Het-Team will perform as well as its homogeneous counterparts, if not better, with Het-Ind being the poor performer.
The intuition is that if there are 2 codependent roles on a team (dropper and collector), they must be assessed together. 
If they are assessed separately, as they would be in Het-Ind, a collector with high reward but low cost (because it never went up the slope) might be selected without its dropper team-mate and will not perform as well when paired with another agent (which is likely to also be a collector).
Good teams will be repeatedly split up and ultimately generalist solutions will be more successful in such an evolutionary environment.
Whereas a Het-Team configuration means that a dropper and collector that work well together will be selected together.
This is the opposite of Waibel et al's study because in a task that doesn't require specialisation, a Het-Ind setup allows multiple genomes to be compared side by side.
That is, two different generalist strategies are compared in parallel during the same experiment.
A Het-Team configuration though, was shown to perform poorly because agents with good strategies are often assessed alongside team-mates with bad strategies, so their entire team is discarded by evolution.
In a task that doesn't require specialisation, team-mates are a hindrance whereas in a task that does require specialisation, they are an asset.
We also suspect that Hom-Ind will achieve similar results to Hom-Team.

\bibliographystyle{plain}
\bibliography{references.bib}

\end{document}